


Für die Vorlage wird das paket \textit{listings} verwendet. \\

\begin{lstlisting} [language=PHP, numbers=left, numberstyle=\tiny, numbersep=10pt]
define('PATH_site', dirname(PATH_thisScript).'/');

if (@is_dir(PATH_site.'typo3/sysext/cms/tslib/')) {
        define('PATH_tslib', PATH_site.'typo3/sysext/cms/tslib/');
} elseif (@is_dir(PATH_site.'tslib/')) {
        define('PATH_tslib', PATH_site.'tslib/');
} else {
      
}
\end{lstlisting}

Quellcodedarstellung 

\begin{verbatim}
user@client:~> texdoc listings
\end{verbatim}

mehr über die Möglichkeiten des Pakets.

------------------------------------------------------\\
\textbf{Notitzen}

20-25 Seiten (Ohne Anhang, inhaltsverzeichniss etc)

Aufgabenstellung

\begin{itemize}
\item[•] Einordnung der Aufgabenstellung in übergeordnete Prozesse/Geschäftsziele
\item[-] Verknüpfung zu Vorlesungsinhalten
\item[-] Praktische Lösung
\item[-] Kritische, inhaltliche Reflexion von Theorie und Praxis
\end{itemize}


------------------------------------------------------\\
\textbf{Form}
Weißes 
Schreibmaschinenpapier  DIN  A  4  (Gewicht  ca.  70  g/qm),  nur  einseitig  beschrieben.  Format:  1,5 
-
zeilig, je Zeile ca. 70 Anschläge, Schriftgröße 12 Punkte
(z.B. bei Arial, b
ei
anderen Schriften ggf. anzupa
s-
sen)
. Randabstand mindestens 2,5 cm allseits.


------------------------------------------------------\\
\textbf{Gliederung der Arbeit}
Die Arbeiten sollten sich in folgende grobe Blöcke unterteilen

\begin{enumerate}
\item Vorspann (Titelblatt, evtl.  Sperrvermerk, Erklärung 
zur Eigenleistung 
[
siehe  2.7
], 
Zusammenfassung
in 
deutsch und englisch, Inhaltsverzeichnis, Abkürzungsverzeichnis, Abbildungs
-
und Tabellen
verzeichnis, 
evtl. Formelgrößen, evtl. Vorwort)

\item Einleitung  (Gegenstand  und  Ziele  der  Arbeit/Aufgabenbeschreibung,  Einführung  in  Thema,  Stand  der 
Technik/Forschung, Motivation der Aufgabenstellung/Vorausblick)
\item 
Hauptteil  (Anforderungsdefinition, 
Anforderungsanalyse,  Lösungsgenerierung,  Lösungsbewertung,  U
m-
setzung) in sinnvollen Gliederungspunkten
\item 
Zusammenfassung und Ausblick
\item 
Literaturverzeichnis
\item 
Anhänge

\end{enumerate}

------------------------------------------------------\\
\textbf{Gestaltung der inhaltlichen Abschnitte/Kapitel}

Einleitung
Die Einleitung soll den Ausgangspunkt der Arbeit umreißen, in kurzer Form zur Problemstellung hinführen 
und das Interesse des Lesers für die Arbeit wecken.
All
gemeine Einleitung ins Thema, keine Unternehmens
-
oder Produktbeschreibungen, Organigramme u.ä., wenn diese nicht direkt zum Thema führen.
Ziele und Vorgehensweise nicht vermischen.
Aufgabenstellung
Die Fragestellung der Aufgabe ist zu präzisieren. 
Insbesondere sind das Umfeld, die vorhandenen Randb
e-
dingungen und Betrachtungsgrenzen darzustellen.
Stand der Technik
Ausgehend von der Aufgabenstellung ist der derzeitige Stand der Technik für die Lösungsfindung zu b
e-
schreiben. Es sind z. B. die Vor
-
und 
Nachteile bisheriger Lösungen bzw. fundamentaler Lösungsprinzipien 
anhand der Literatur darzulegen.
Hauptteil
Der Text soll knapp und klar sein und die wesentlichen Gedanken der Arbeit beinhalten. Ein gewähltes Ve
r-
fahren oder ein bestimmter Lösungsweg muss
begründet werden. Es ist nicht notwendig, alle Vorversuche 
einzeln zu schildern. Bei Versuchen sind Voraussetzungen und Vernachlässigungen sowie die Anordnung, 
Leistungsfähigkeit und Messgenauigkeit der Versuchsanordnung anzugeben.
Die Ergebnisse der Arbe
it sind unter Berücksichtigung der Voraussetzungen ausführlich zu diskutieren und 
mit den bereits bekannten Anschauungen und Erfahrungen zu vergleichen.
Ziel der Arbeit ist es, eindeutige Folgerungen und Richtlinien für die Praxis zu finden.
Zusammenfassun
g 
Aufgabenstellung, Vorgehensweise und wesentliche Ergebnisse werden kurz und präzise dargestellt. 
Die 
Zusammenfassung ist eigenständig verständlich. Länge ca. 1 bis 1,5 Seiten.(Problem, Ziele, Vorgehenswe
i-
se, Ergebnisse und Ausblick).