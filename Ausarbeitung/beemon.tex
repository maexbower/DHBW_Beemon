% Präambel
\documentclass[12pt,a4paper,oneside, 
    listof=totoc, 					% Tabellen- und Abbildungsverzeichnis ins Inhaltsverzeichnis
    bibliography=totoc,						% Literaturverzeichnis ins Inhaltsverzeichnis aufnehmen
    titlepage, 						% Titlepage-Umgebung statt \maketitle
    headsepline, 					% horizontale Linie unter Kolumnentitel
    %abstracton,					% Überschrift beim Abstract einschalten, Abstract muss dazu in {abstract}-Umgebung stehen
    %DIV11,							% auskommentieren, um den Seitenspiegel zu vergrößern
    BCOR6mm,						% Bindekorrektur, die den Seitenspiegel um 6mm nach rechts verschiebt,
]{scrreprt}			
\usepackage{ucs} 				% Dokument in utf8-Codierung schreiben und speichern
\usepackage[utf8x]{inputenc} 	% ermöglicht die direkte Eingabe von Umlauten
\usepackage[ngerman]{babel} 	% deutsche Trennungsregeln und Übersetzung der festcodierten Überschriften
\usepackage[T1]{fontenc} 		% Ausgabe aller zeichen in einer T1-Codierung (wichtig für die Ausgabe von Umlauten!)
\usepackage{graphicx}  			% Einbinden von Grafiken erlauben
\usepackage{amsmath}
\usepackage{amsfonts}
\usepackage{amssymb}
\usepackage{mathpazo} 			% Einstellung der verwendeten Schriftarten
\usepackage{textcomp} 			% zum Einsatz von Eurozeichen u. a. Symbolen
\usepackage{listings}			% Datstellung von Quellcode mit den Umgebungen {lstlisting}, \lstinline und \lstinputlisting
\usepackage{xcolor} 			% einfache Verwendung von Farben in nahezu allen Farbmodellen
\usepackage[intoc]{nomencl}  	% zur Erstellung des Abkürzungsberzeichnisses
\usepackage{fancyhdr}			% Zusatzpaket zur Gestaltung von Fuß und Kopfzeilen
\usepackage{titleref}			% Zum referenzieren mit Überschrift
\usepackage{varwidth}			%alighn asci figure
\usepackage{chngcntr}			% Tabellen- und Abbildungsverzeichnis als fortlaufende Nummer
\usepackage{hyperref}
\usepackage[justification=centering]{caption} % Bild unterschrift mittig ausrichten
\usepackage{nameref}
\counterwithout{figure}{chapter}
\counterwithout{table}{chapter}
\usepackage{url}
\usepackage{ulem}


%For flow chart
\usepackage{tikz}
\usetikzlibrary{shapes,arrows}
% Define block styles
\tikzstyle{decision} = [diamond, draw, fill=blue!20, 
    text width=4.5em, text badly centered, node distance=3cm, inner sep=0pt]
\tikzstyle{block} = [rectangle, draw, fill=blue!20, 
    text width=5em, text centered, rounded corners, minimum height=4em, node distance=3cm]
\tikzstyle{line} = [draw, -latex']
\tikzstyle{cloud} = [draw, ellipse,fill=red!20, node distance=2cm,
    minimum height=2em]


%\usepackage[none]{hyphenat} %deaktiviert Silbentrennung
%\usepackage{showframe}% zum Anzeigen des Seitenlayouts
%Abstand vor Chapter
\renewcommand*\chapterheadstartvskip{\vspace*{-\topskip}}
\renewcommand*\chapterheadendvskip{%
  \vspace*{1\baselineskip plus .1\baselineskip minus .167\baselineskip}}


\setlength\abovedisplayshortskip{0pt}
\setlength\belowdisplayshortskip{0pt}
\setlength\abovedisplayskip{20pt}
\setlength\belowdisplayskip{20pt}

% -----------------------------------------------------------------------------------------------------------------
% Zum Aktualisieren des Abkürzungsverzeichnisses bitte auf der Kommandozeile folgenden Befehl aufrufen :
%  makeindex Bachelorarbeit.nlo -s nomencl.ist -o Bachelorarbeit.nls
% -----------------------------------------------------------------------------------------------------------------

% Hier die persönlichen Daten eingeben:

\newcommand{\titel}{Titel}
\newcommand{\untertitel}{Projekt untertitel}
\newcommand{\arbeit}{Projektarbeit}
\newcommand{\studiengang}{Informatik}
\newcommand{\autor}{Maximilian Bauknracht\\ Nina Zaske\\ Philippe Käufer}
\newcommand{\kurs}{TINF15K}
\newcommand{\abgabe}{\today}
\newcommand{\betreuerdhbw}{Maximilian Scholl}

\newcommand{\jahr}{2017 - 2018}			% für Angabe im Copyright-Vermerk der Titelseite

% Abkürzungen
\newcommand{\ua}{\mbox{u.\,a.\ }}
\newcommand{\zB}{\mbox{z.B.\ }}
\newcommand{\bs}{$\backslash$}
\newcommand{\Csharp}{C\#}

\renewcommand{\nomname}{Abkürzungsverzeichnis}
\makenomenclature 

% -------------------------------------------------------------------------------------------
% Definition der Kopf- und Fußzeilen
\lhead{}								% Kopf links
\chead{}								% Kopf mitte
\rhead{\sffamily{\titel}}				% Kopf rechts
\lfoot{}								% Fuß links
\cfoot{\sffamily{\thepage}}				% Fuß mitte
\rfoot{\sffamily{\autor}}				% Fuß rechts
\renewcommand{\headrulewidth}{0.4pt}	% Liniendicke Kopf
\renewcommand{\footrulewidth}{0.4pt}	% Liniendicke Fuß

\makenomenclature						% Abkürzungsverzeichnis erstellen

% alle Abkürzungen, die in der Arbeit verwendet werden

\nomenclature{DHBW}{Duale Hochschule Baden-Württemberg}
				% Datei mit Abkürzungen laden

%%makeindex t2000.nlo -s nomencl.ist -o t2000.nls %% Abkürzungsverzeichnis erstellen

% -------------------------------------------------------------------------------------------
%                     Beginn des Dokumenteninhalts
% -------------------------------------------------------------------------------------------
\begin{document}
\setcounter{secnumdepth}{2}					% Nummerierungstiefe fürs Inhaltsverzeichnis
\setcounter{tocdepth}{2}
\sffamily									% für die Titelei serifenlose Schrift verwenden

% ------------------------------ Titelei -----------------------------------------------------

\thispagestyle{plain}
\begin{titlepage}
\enlargethispage{4.0cm}
%\sffamily 								% Serifenlose Grundschrift für die Titelseite einstellen
\begin{figure}[!htb]
\hfill
\minipage{0.4\textwidth}%
  \includegraphics[width=\linewidth]{Bilder/logo_dhbw.jpg}
\endminipage
\end{figure}

\begin{center}

\huge{\textsc{\textbf{\titel}}}\\[1.5ex]
\Large{\textbf{\untertitel}}\\[5ex]
\LARGE{\textbf{\arbeit}}\\[2ex]
\Large{Studiengang \studiengang}\\[1ex]
\normalsize{Bachelor of Science}\\[1ex]
\normalsize{Duale Hochschule Baden-Württemberg Stuttgart}\\[3ex]

von\\[1ex] \LARGE{\autor} \\[5ex]


\end{center}

\begin{flushleft}

\begin{tabular}{ll}
Abgabedatum:					& \quad \abgabe \\
%Bearbeitungszeitraum:			& \quad 12 Wochen   \\ 
Kurs: 			& \quad \kurs \\ 
Betreuer der Dualen Hochschule: & \quad \betreuerdhbw \\ [5ex]

\end{tabular} 



\small
Copyrightvermerk:\\

Dieses Werk einschließlich seiner Teile ist \textbf{urheberrechtlich geschützt}. Jede Verwertung außerhalb der engen Grenzen des Urheberrechtgesetzes ist ohne Zustimmung des Autors unzulässig und strafbar. Das gilt insbesondere für Vervielfältigungen, Übersetzungen, Mikroverfilmungen sowie die Einspeicherung und Verarbeitung in elektronischen Systemen.
\end{flushleft}
\begin{flushright}
\copyright{} \jahr
\end{flushright}
\end{titlepage}
\rmfamily 				% erzeugt die Titelseite
\pagenumbering{Roman}						% große, römische Seitenzahlen für Titelei
\addchap{Eidesstattliche Erklärung}
Ich versichere hiermit, dass ich meine Praxisarbeit mit dem Thema
\begin{quote}
\textit{\titel} -\textit{ \untertitel }
\end{quote}
selbständig verfasst und keine anderen als die angegebenen Quellen und Hilfsmittel benutzt habe. Die Arbeit wurde bisher keiner anderen Prüfungsbehörde vorgelegt und auch nicht veröffentlicht. \\Die Abgabe der Dokumentation erfolgt gedruckt im Sekretariat der Dualen Hochschule Baden-Württemberg Stuttgart.\\[10ex]

Stuttgart, den \today \\[4ex]


%\rule[-0.2cm]{5cm}{0.5pt} \\	
\begin{figure}[h]
%\includegraphics{Bilder/Unterschrift2.jpg}
\end{figure}
\textsc{\autor} \\[10ex]
 				% Einbinden der eidestattlichen Erklärung
\chapter*{Projektbeschreibung} %*-Variante sorgt dafür, das Abstract nicht im Inhaltsverzeichnis auftaucht


\section{Titel}
   				% Einbinden des Abstracts

\tableofcontents							% Erzeugen des Inhalsverzeichnisses
\printnomenclature[2.0cm]					% Erzeugen des Abkürzungsverzeichnisses
\listoffigures 								% Erzeugen des Abbildungsverzeichnisses 
\listoftables 								% Erzeugen des Tabellenverzeichnisses
\pagebreak


% --------------------------------------------------------------------------------------------
%                    Inhalt der Bachelorarbeit
%---------------------------------------------------------------------------------------------
\pagenumbering{arabic}						% arabische Seitenzahlen für den Hauptteil
\pagestyle{fancy}					
\rmfamily
\noindent

\chapter{Einführung}
\label{cha:Einleitung}


\section{Motivation}
\label{sec:Motivation}

\section{Projektumfeld}
\label{sec:Projektumfeld}

\section{Ziel der Arbeit}
\label{sec:ZielDerArbeit}

\chapter{Text}
\label{cha:text}
\chapter{Reflektion}
\label{cha:reflektion}

\chapter{Ausblick}
\label{cha:ausblick}

%\include{Inhalt/vorlagen}


% ---------------------------- Literaturverzeichnis ----------------------------------------------

\begin{thebibliography}{999999}

%\bibitem[Le01]{levy2001} Autor Name: \emph{Titel des Buches}, New York: Penguin Books, 2001


%\bibitem {Buchtitel} Author: \emph{Buchtitel}, Erscheinungsort: Verlag, Jahr


\end{thebibliography}

% ------------------------------- Anhang ---------------------------------------------------------

\begin{appendix}
\clearpage
\pagenumbering{Roman}						% römische Seitenzahlen für Anhang
\end{appendix}


\end{document}
